%%%%%%%%%%%%%%%%%%%%%%%%%%%%%%%%%%%%%%%%%%%%%%%%%%%%%%%%%%%%%%%%%%%%%%%%%%%%%%%%
% TITOLO
%%%%%%%%%%%%%%%%%%%%%%%%%%%%%%%%%%%%%%%%%%%%%%%%%%%%%%%%%%%%%%%%%%%%%%%%%%%%%%%%
\newcommand{\image}[3]{ % 1 image 2 caption 3 size
	\begin{figure}[h!]
		\centering
		\includegraphics[width=#3\textwidth]{#1} 
		\caption{#2}
	\end{figure}
	\FloatBarrier
}

\newcommand{\R}{{\rm I\!R}}

\newcommand{\imageLabel}[4]{ % 1 image 2 caption 3 size
	\begin{figure}[h!]
		\centering
		\includegraphics[width=#3\textwidth]{#1} 
		\caption{#2}
		\label{fig:#4}
	\end{figure}
	\FloatBarrier
}
\newcommand{\Z}{\mathbb{Z}}

\pagenumbering{Alph}
\begin{titlepage}
	\begin{center}
		\includegraphics[width=0.7\textwidth]{unive}
		
		\vspace*{1cm}
		\LARGE
		\textit{Artificial Intelligence: knowledge representation and planning\\ \center Year: 2017/2018}
		
		\vspace{0.5cm}
		\Huge
		\textbf{\titolo}\\
		\LARGE {\nome}
		
		\line(1,0){280}
		
		\vspace{0.5cm}
		\large
		

		\Large
		\Large Author: \textbf{Antonio Emanuele Cinà} \\
		\vspace{0.5cm}

		\textit{\today }
		
		\vfill
		
	\end{center}
\end{titlepage}

%%%%%%%%%%%%%%%%%%%%%%%%%%%%%%%%%%%%%%%%%%%%%%%%%%%%%%%%%%%%%%%%%%%%%%%%%%%%%%%%
%% STILE HEADER - FOOTER - LISTE
%%%%%%%%%%%%%%%%%%%%%%%%%%%%%%%%%%%%%%%%%%%%%%%%%%%%%%%%%%%%%%%%%%%%%%%%%%%%%%%%

\renewcommand{\headheight}{14pt}

\pagestyle{fancy}
\lhead{}
\chead{}
\lhead{\textit{Author: acina}}
\rhead{\textbf{\titolo}}
\cfoot{}
\renewcommand{\headrulewidth}{0.4pt}
\renewcommand{\footrulewidth}{0.4pt}

%\renewcommand{\labelitemi}{$\diamond$}
%\renewcommand{\labelitemii}{$\bullet$}
\renewcommand{\labelitemi}{$\bullet$}
\renewcommand{\labelitemii}{$\diamond$}
\renewcommand{\labelitemiii}{$\circ$}

\setlist{itemsep=0pt}

\setlength{\parindent}{0cm}

%%%%%%%%%%%%%%%%%%%%%%%%%%%%%%%%%%%%%%%%%%%%%%%%%%%%%%%%%%%%%%%%%%%%%%%%%%%%%%%%
%% INDICE
%%%%%%%%%%%%%%%%%%%%%%%%%%%%%%%%%%%%%%%%%%%%%%%%%%%%%%%%%%%%%%%%%%%%%%%%%%%%%%%%

\pagenumbering{gobble}
\renewcommand{\contentsname}{Index}
\tableofcontents
\newpage
\pagenumbering{arabic}

%%%%%%%%%%%%%%%%%%%%%%%%%%%%%%%%%%%%%%%%%%%%%%%%%%%%%%%%%%%%%%%%%%%%%%%%%%%%%%%%
%% FOOTER CON NUMERO PAGINA
%%%%%%%%%%%%%%%%%%%%%%%%%%%%%%%%%%%%%%%%%%%%%%%%%%%%%%%%%%%%%%%%%%%%%%%%%%%%%%%%

\rfoot{\thepage\ di \pageref{LastPage}}



\definecolor{mygreen}{rgb}{0,0.6,0}
\definecolor{mygray}{rgb}{0.5,0.5,0.5}
\definecolor{mymauve}{rgb}{0.58,0,0.82}

\lstset{ %
	backgroundcolor=\color{white},   % choose the background color; you must add \usepackage{color} or \usepackage{xcolor}; should come as last argument
	basicstyle=\footnotesize,        % the size of the fonts that are used for the code
	breakatwhitespace=false,         % sets if automatic breaks should only happen at whitespace
	breaklines=true,                 % sets automatic line breaking
	captionpos=b,                    % sets the caption-position to bottom
	commentstyle=\color{mygreen},    % comment style
	deletekeywords={...},            % if you want to delete keywords from the given language
	escapeinside={\%*}{*)},          % if you want to add LaTeX within your code
	extendedchars=true,              % lets you use non-ASCII characters; for 8-bits encodings only, does not work with UTF-8
	frame=single,	                   % adds a frame around the code
	keepspaces=true,                 % keeps spaces in text, useful for keeping indentation of code (possibly needs columns=flexible)
	keywordstyle=\color{blue},       % keyword style
	language=Octave,                 % the language of the code
	morekeywords={*,...},            % if you want to add more keywords to the set
	numbers=left,                    % where to put the line-numbers; possible values are (none, left, right)
	numbersep=5pt,                   % how far the line-numbers are from the code
	numberstyle=\tiny\color{mygray}, % the style that is used for the line-numbers
	rulecolor=\color{black},         % if not set, the frame-color may be changed on line-breaks within not-black text (e.g. comments (green here))
	showspaces=false,                % show spaces everywhere adding particular underscores; it overrides 'showstringspaces'
	showstringspaces=false,          % underline spaces within strings only
	showtabs=false,                  % show tabs within strings adding particular underscores
	stepnumber=2,                    % the step between two line-numbers. If it's 1, each line will be numbered
	stringstyle=\color{mymauve},     % string literal style
	tabsize=2,	                   % sets default tabsize to 2 spaces
	title=\lstname                   % show the filename of files included with \lstinputlisting; also try caption instead of title
}

\lstset{
	language=Python,
	basicstyle=\ttfamily,
	otherkeywords={self},             
	keywordstyle=\ttfamily\color{blue!90!black},
	keywords=[2]{True,False,reshape},
	keywords=[3]{ttkc},
	keywordstyle={[2]\ttfamily\color{orange}},
	keywordstyle={[3]\ttfamily\color{red!80!orange}},
	emph={MyClass,__init__, False, True,dot,reshape,SVC,cross_val_score},          
	emphstyle=\ttfamily\color{red!80!black},    
	stringstyle=\color{cyan!80!black},
	showstringspaces=false            
}


%\begin{table}[!htbp]
%	\begin{tabularx}{\linewidth}{|Y|l|l|S{Z}|l}
%		\cline{1-4}
%		\thead{Risk Event} & \thead{Chance\\ of Happening} & \thead{Severity} & \thead{Measures\\ to be taken} & \\
%		\cline{1-4}
%		Team member\break missing meetings & Significant & Low & Encourage team members to read over minutes and inform of any tasks set. If regularly absent issue a warning and then card. & \\
%		\cline{1-4}
%		QA/Project manager missing meetings & Significant & Low/Moderate & Deputy in role will act as manager. & \\
%		\cline{1-4}
%		Team member leaving project group & Low & Moderate/High & List all tasks assigned to missing team member and reassign them, after this refactor the timetable and planning. & \\
%		\cline{1-4}
%	\end{tabularx}%
%\end{table}